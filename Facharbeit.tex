\documentclass[a4paper, 12pt]{report}
\usepackage[german]{babel}
\title{Facharbeit Informatik}
\author{Joel Mantik}
\begin{document}
\maketitle
\tableofcontents

\chapter{Einleitung}
Der Gauß-Algorithmus ist eines der wichtigsten Lösungsverfahren zum Lösen linearer Gleichungssysteme.
Es spielt eine tragende Rolle in der in vielen Bereichen der Mathematik und ist dennoch recht unkompliziert
in der Ausführung. Aufgrund dieser Tatsachen habe ich mich dazu entschieden, in dieser Facharbeit den Algorithmus in verschiedener Weise zu implementieren, ihn zu analysieren,
die verschieden Methoden zu vergleichen, Anwendungsmöglichkeiten aufzuzeigen und schlussendlich ein Fazit zu ziehen.

\chapter{Theoretische Grundlagen}
\section{Lineare Gleichungssysteme}
Um zu verstehen, wie das Gaußschen-Eliminationsverfahren implementiert werden kann, wird sich erst einmal
die Definition linearer Gleichungssysteme angeschaut.

\newline
\textbf{Definition:}
Ein lineares Gleichungssystem ist in der linearen Algebra eine Menge linearer Gleichungen mit einer oder mehreren
Unbekannten, die alle gleichzeitig erfüllt sein sollen.
\newline
Ein lineares Gleichungssystem hat folgende Form:
\begin{eqnarray}
    a_{11}x_{1}+ a_{12}x_{2}+...+ a_{1n}x_{n}     & = &  b_1\\
    a_{21}x_{1}+ a_{22}x_{2}+...+ a_{2n}x_{n}     & = &  b_2\\
    a_{11}x_{1}+ a_{12}x_{2}+...+ a_{1n}x_{n}     & = &  b_3\\
\end{eqnarray}
\section{Gaußsches Eliminationsverfahren}
\chapter{Implementierung des Algorithmus}
\section{Schritte des Verfahrens}
\subsection{Brute-Force Methode}
\subsection{Pivot-Suche}
\section{Vergleich der Methoden}
\chapter{Anwendungen des Gaußschen-Eliminationsverfahrens}
\section{Lösung von linearen Gleichungssystemen}
\section{Inversion von Matrizen}
\section{Beispielanwendungen}
\chapter{Abwägungen}
\section{Vor- und Nachteile im Vergleich zu anderen Methoden}
\chapter{Fazit}
\end{document}
