\documentclass[a4paper, 12pt]{report}
\usepackage[ngerman]{babel}
\usepackage{amssymb}
\usepackage{tabularx}
\usepackage[utf8]{inputenc}
\usepackage{amsmath}
\usepackage{tikz-uml}
\usepackage{bbm}
\usepackage{titlesec}
\titleformat{\chapter}[display]
    {\normalfont\LARGE\bfseries}{\chaptertitlename\ \thechapter}{20pt}{\LARGE}
\usepackage{setspace}
\usepackage{fancyvrb}
\usepackage{graphicx}
\usepackage[style=alphabetic ,backend=biber]{biblatex}
\usepackage{csquotes}
\usepackage{array}
\usepackage{geometry}
\geometry{
 a4paper,
 total={170mm,257mm},
 left=25mm,
 right=40mm,
 top=25mm,
 bottom=40mm
 }
 \setstretch{1.5}
\addbibresource{../Facharbeit/literatur.bib}
\usepackage{listings}
\usepackage{xcolor}
\tikzumlset{fill class=white}
\definecolor{codegreen}{rgb}{0,0.6,0}
\definecolor{codegray}{rgb}{0.5,0.5,0.5}
\definecolor{codepurple}{rgb}{0.58,0,0.82}
\definecolor{backcolour}{rgb}{255,255,255}

\newcommand{\GE}{Gauß'sches Eliminationsverfahren }
\newcommand{\GA}{Gauß-Algorithmus }
\lstdefinestyle{codehighligthing}{
    backgroundcolor=\color{backcolour},
    commentstyle=\color{codegreen},
    keywordstyle=\color{magenta},
    numberstyle=\tiny\color{codegray},
    stringstyle=\color{codepurple},
    basicstyle=\ttfamily\tiny,
    breakatwhitespace=false,
    breaklines=true,
    captionpos=b,
    keepspaces=true,
    numbers=left,
    numbersep=5pt,
    showspaces=false,
    showstringspaces=false,
    showtabs=false,
    tabsize=2,
}
\lstset{style=codehighligthing}


\title{\LARGE Facharbeit im Grundkurs Informatik  \vspace*{0.5cm} \\
  \Huge Implementation des \GA \vspace*{0.5cm} \\
  \LARGE Georg-Büchner-Gymnasium}
\LARGE \author{Joel Mantik
\vspace*{0.5cm} \\ Pablo Sonnauer}
\LARGE \date{März 2023}
\begin{document}
\maketitle
\tableofcontents

{\let\clearpage\relax \chapter{Einleitung}}
\begin{quote}
    "The simplest model in applied mathematics is a system of linear equations.
    It is also by far the most important." \\ - GILBERT STRANG
\end{quote}

Die lineare Algebra stellt eines der wichtigsten Felder der Mathematik dar. Wie aus dem Zitat des Mathematikers Gilbert Strang hervorgeht,
sind die linearen Gleichungssysteme trotz ihrer Simplizität eines der wichtigsten Konzepte in der angewandten Mathematik.
Eines der faszinierendsten Lösungsverfahren dieser Gleichungssysteme ist der Gauß-Algorithmus,
da er das Lösen auf einfachste Weise möglich macht. Seit seiner Entdeckung im frühen 19. Jahrhundert durch den
Mathematiker Carl Friedrich Gauß stellt er eines der elementarsten Lösungsverfahren Linearer Gleichungssysteme dar.
Auch in der Informatik spielt der Algorithmus eine große Rolle, beispielsweise in der Computergrafik. Aus diesen Gründen wird im Folgenden eine Implementationsmöglichkeit
vorgestellt. Zuerst wird der mathematische Hintergrund erläutert, anschließend der Quellcode des Algorithmus vorgestellt und
Vor- und Nachteile abgewogen.
Weiterhin werden spezifische Anwendungsmöglichkeiten in der Informatik aufgezeigt und zuletzt wird ein Fazit gezogen.

\chapter{Lineare Gleichungssysteme}

Ein lineares Gleichungssystem ist eine Sammlung von Gleichungen, in denen jede Unbekannte mit höchstens
dem ersten Grad vorkommt.
Es kann in der Form $ A_x = b $ geschrieben werden, wobei $A$ eine Matrix der Form $ m * n$,
$x$ ein n-dimensionaler Vektor von Unbekannten
und b ein $m$-dimensionaler Vektor von Konstanten sei.
Ein allgemeines lineares Gleichungssystem lässt sich wie folgt definieren.: \cite{1}
\begin{align}
    \label{eq:linGle}
    a_{11}x_{1}+ a_{12}x_{2}+\hdots+ a_{1n}x_{n} &= b_1   \nonumber \\
    a_{21}x_{1}+ a_{22}x_{2}+\hdots+ a_{2n}x_{n} &= b_2 \nonumber\\
                                                 &\vdots   \nonumber \\
    a_{m1}x_{1}+ a_{m2}x_2+\hdots+a_{mn}x_{n} &= b_{m}
\end{align}
Das Ziel eines solchen linearen Gleichungssystems ist es, eine Lösung für $x$ zu finden, die alle Gleichungen erfüllt.
Hierbei gibt es drei Arten von Lösungen:
\begin{enumerate}
    \item Das Gleichungssystem hat genau \textit{eine} Lösung;
        es gibt genau eine Lösung, welche alle Gleichungen im System
        erfüllt. Die Lösungsmenge ist zum Beispiel: $\mathbb{L} = \{ (x,y,z)| (1,2,3)\} $.
    \item Das Gleichungssystem hat \textit{keine} Lösung, wenn es keine Lösung gibt,
        die alle Gleichungen erfüllt. Die Lösungsmenge ist eine leere Menge: $ \mathbb{L}= \emptyset$.
    \item Das Gleichungssystem hat \textit{unendlich viele} Lösungen, wenn es mehrere Lösungen gibt,
        die alle Gleichungen im System erfüllen.
        Hierbei sind die verschiedenen Variablen möglicherweise voneinander abhängig.
        Die Lösungsmenge sieht beispielsweise wie folgt aus:
        \newline $ \mathbb{L} = \{(x, y, z)| (x = ay + z, y\in \mathbb{R}, z \in \mathbb{R}) \} $.
\end{enumerate}


{\let\clearpage\relax \chapter{\GE}} \label{2.2}
Gegeben sei das Allgemeine lineare Gleichungssystem \ref{eq:linGle}.
Gesucht ist nun die Menge der $ (x_1, \hdots ,x_n ) \in \mathbb{R}^n $, die alle Gleichungen erfüllen.
Um diese Menge zu finden, wird folgendermaßen vorgegangen:
\begin{enumerate}
    \item Das Gleichungssystem in die erweiterte Koeffizientenmatrix $ (A, b) $ umschreiben, \cite{2}:
        \begin{equation}
            (A, b):=
            \begin{pmatrix}
                a_{11} & \hdots &  a_{1n} &  b_1  \\
                \vdots & \vdots &  \vdots & \vdots &  \\
                a_{m1} &  \hdots &  a_{mn} &  b_m \\
            \end{pmatrix}
        \end{equation}
    \item $(A, b)$ durch elementare Zeilentransformationen, also Vertauschen von Zeilen,
        Multiplikation einer Zeile mit einer Zahl $\neq 0 $ oder Addition des Vielfachen von einer Zeile zu einer
        anderen Zeile auf Zeilenstufenform bringen.
        \begin{center}
            \begin{equation}
                \label{Zeilenstufenform}
                \left(\begin{array}{cccccc}
                        1 & a_{12} & a_{13} & \cdots & a_{1,n-1} & a_{1n} \\
                        0 & 1 & a_{23} & \cdots & a_{2,n-1} & a_{2n} \\
                        0 & 0 & 1 & \cdots & a_{3,n-1} & a_{3n} \\
                        \vdots & \vdots & \vdots & \ddots & \vdots & \vdots \\
                        0 & 0 & 0 & \cdots & 1 & a_{n-1,n} \\
                \end{array}\right)
            \end{equation}
        \end{center}

        Eine $ m * n $-Matrix $(A,b)$, wie in \ref{Zeilenstufenform} wird  \textit{Zeilenstufenform} genannt.
        Das \textit{a} steht für eine beliebige reelle Zahl, alle anderen Plätze ohne ein \textit{a} sind von Nullen besetzt.
        Der erste von null verschiedene Eintrag in jeder Zeile ist eins. Dieser Eintrag wird das Pivot-Element der Zeile genannt. \\
        Das Pivot-Element der $(i + 1)$-ten Zeile steht immer rechts des Pivot-Elements der I-ten Zeile, alle Einträge
        oberhalb eines Pivot-Elements sind gleich null. Vgl. \cite{5}.
    \item Nun lassen sich durch Rücksubstitution die Werte für die Variablen ermitteln.
        Es wird die letzte Spalte durch den Wert des Koeffizienten geteilt, sodass die Variable alleine steht.
        Der gefundene Wert wird dann in die nächst höhere Zeile eingesetzt, anschließend wird analog zum ersten Schritt vorgegangen.
\end{enumerate}

{\let\clearpage\relax \chapter{Erläuterung des Quellcodes}}
Für die im Folgenden dargelegte Implementation des Algorithmus wurde die Programmiersprache \texttt{"Java"} verwendet. \newpage

\begin{figure}
    \label{IMpl}
\begin{center}
{\ttfamily
\begin{tikzpicture}[scale=0.6]
    \tiny
  \umlclass{Gauß}{
    - datei: String \\
    - countSpalten: int \\
    - NORMALFALL: boolean \\
    - koeff: double[][] \\
    - solMatrix: double[][] \\
    - zeilenElemente: String[] \\
    - countZeilen: int \\
    - SONDERFALL: boolean
  }{
    - main(args: String[]): void \\
    - einlesen(): void \\
    - ausgabe(): void \\
    - gaussAlgo(): void \\
    - multiplyAndAdd(lineOne: int, lineTwo: int, \\ factor: double):void \\
    - checkWievielNullZeilen(): boolean \\
    - checkObNull(): boolean
  }
\end{tikzpicture}
}
\caption{Implementationsdiagramm des \GA}
\end{center}
\end{figure}
Das obige Implementationsdiagramm zeigt den grundlegenden Aufbau.
Es wurden aussagefähige Methodennamen gewählt.
In der Main Methode des Programms wird ein Parameter übernommen,
welcher die Datei mit der Koeffizientenmatrix annimmt. Außerdem werden
in dieser die Methoden \texttt{einlesen(), ausgabe() und gaussAlgo()} ausgeführt.
Die Methode \texttt{ausgabe()} gibt in der Main Methode die Eingangsmatrix, die triangularisierte Matrix
und die Lösungs-\\matrix aus. \newline
Die Methode \texttt{einlesen()} liest die Koeffizientenmatrix aus der Datei ein,
dabei wird eine Liste erstellt, in welcher alle Zeilen der eingelesenen Datei gespeichert werden.
Die Datei kann Kommentare und Leerzeilen enthalten, daher werden durch eine For-Schleife alle Leerzeichen und Kommentare entfernt.
Des Weiteren werden die Kommata in Punkte umgewandelt,
um das Funktionieren des Algorithmus auch auf anderen Systemen zu gewährleisten.
Weiterhin wird die Anzahl der Spalten in der Variable \texttt{countSpalten},
und die Anzahl der Zeilen in \texttt{countZeilen} gespeichert und
im Zuge dessen werden auch die \textit{Eingangsmatrix} \texttt{koeff[][]}und die \textit{Lösungsmatrix} \texttt{solMatrix[][]} initialisiert.
Die Methode \texttt{ausgabe()} nimmt zwei Parameter an, das zweidimensionale Array \texttt{double mx}
und den String \texttt{matrixName}. Beide werden durch eine For-Schleife auf der Konsole ausgegeben.
Die Methode \texttt{multiplyAndAdd()}, welche die Parameter \texttt{lineOne, lineTwo (Integer)}
und den \texttt{double factor} annimmt, wird verwendet, um die Koeffizienten auf den Wert null zu bringen.
Dabei wird durch eine For-Schleife über die Spalten der Matrix iteriert
und dabei der Koeffizient an der Stelle von \texttt{lineTwo} und \texttt{spalten} auf den Wert null gebracht.
Dies geschieht, indem die Matrix an der Stelle \texttt{lineOne} und \texttt{spalten} mit dem Parameter \texttt{factor} multipliziert
und zu dem zu verändernden Koeffizienten addiert wird.
Weiter gibt es die Methode \texttt{checkObNull()}, welche den Parameter \texttt{zeile (Integer)} annimmt
und über die Eingangsmartix iteriert und prüft, ob ein Koeffizient den Wert null hat. In diesem Fall wird
\texttt{true} zurückgegeben, andernfalls \texttt{false}.
Diese Methode wird später für die Sonderbehandlung von Matrizen verwendet, die entweder eine Null-Zeile haben oder gleich null sind.
Anknüpfend gibt es noch die Methode \texttt{checkWievielNullZeilen()}. Diese iteriert, ausgehend von der letzten Zeile,
über alle Zeilen und prüft, ob alle Koeffizienten in der Zeile gleich null sind. Für jede gefundene Null-Zeile wird ein Counter
\texttt{nullCounter} erhöht. Wenn diese Zählvariable am Ende der Iteration ungleich Null ist, wird die Konstante \texttt{SONDERFALL} zurückgegeben
und die Anzahl der Null-Zeilen auf der Konsole ausgegeben, andernfalls wird die Konstante \texttt{NORMALFALL} zurückgegeben.
Wenn die Matrix an jeder Stelle eine Null hat, wird ebenfalls interveniert und der Sonderfall auf der Konsole ausgegeben.
Die Methode arbeitet sehr effektiv, denn durch die Pivotisierung kann nach erfolgter Triangularisierung von der letzten Zeile aus iteriert werden, denn
wenn es Null-Zeilen gibt, so sind diese am Ende.

Die Methode \texttt{gaussAlgo()} führt den eigentlichen Algorithmus aus.
Als Erstes wird die Eingangsmartix, also \texttt{koeff[][]}, durch eine For-Schleife in die Lösungsmatrix \texttt{solMatrix[][]} kopiert, damit am Ende
beide ausgegeben werden können.
Anschließend findet die Pivotisierung der Matrix statt. Dies geschieht, indem zuerst
eine Variable \texttt{maxZeile (Integer)} erzeugt und mit der Zählvariable \texttt{zeile} initialisiert wird.
Diese wird verwendet, um die Zeile mit dem größten Koeffizienten zu finden.
In der darauf folgenden For-Schleife findet die eigentliche Pivotisierung statt.
Sie sucht in der aktuellen Spalte nach dem Element mit dem größten, absoluten Wert
und speichert diesen, wenn der größere Wert in der Zeile unter der aktuellen Zeile ist, in der Variable \texttt{maxZeile}.
Anschließend wird geprüft, ob die zuvor initialisierte maxZeile ungleich dem Wert der Variable \texttt{zeile} ist.
Wenn dem so ist, werden die Zeilen mithilfe einer temporären Variable getauscht, sodass am Ende die
Zahl mit dem größten Koeffizienten oben steht.
Als Nächstes wird der Faktor berechnet, mit dem die Koeffizienten gleich null werden.
Dies geschieht durch die nächste For-Schleife, welche den Faktor \texttt{factor} berechnet, indem die Lösungsmatrix mit negativem Vorzeichen an der Stelle der Zählvariable i
und \texttt{zeile} (die Zählvariable der ersten for-Schleife), durch die Lösungsmatrix an der Stelle \texttt{zeile, zeile} geteilt wird.
Dieser Faktor wird dann an die Methode \texttt{multiplyAndAdd} zusammen mit \texttt{zeile, i} als Parameter übergeben. Nachdem
die For-Schleife durchgelaufen ist, wird die entstandene triangularisierte Matrix  ausgegeben. \newline Anschließend prüft der Algorithmus
die triangularisierte Matrix auf Sonderfälle, also ob die Matrix Null-Zeilen hat oder komplett gleich Null ist, indem durch eine Wenn-dann prüfung
abgefragt wird, ob die Methode \texttt{checkWievielNullZeilen} einen Sonderfall zurückgibt. Wenn sie dies tut, bricht der Algorithmus ab. \newline
Der letzte Schritt wird verwendet, um die Matrix durch Rücksubstitution zu lösen.
Eine For-Schleife läuft von der letzten bis zur ersten Spalte. Zuerst wird in dieser Schleife eine Variable \texttt{double diagonale}
erstellt, in welcher das aktuelle Diagonalelement gespeichert wird. In der nächsten Zeile iteriert eine weitere For-Schleife
über die Matrix und teilt bei jeder Iteration den Wert des aktuellen Koeffizienten durch den Wert in \texttt{diagonale}.
Dadurch entsteht die Dreiecksform, bei welcher jedes Element in dieser den Wert Eins haben soll.
Weitergehend wird der Wert des Tupels in der Variable \texttt{double result} gespeichert.
Durch eine weitere For-Schleife wird die über alle Zeilen unter der aktuellen iteriert, bei jeder Iteration wird der Wert
der rechten Zeile um das Produkt aus dem Wert von \texttt{result} und dem Wert des Elements aus der aktuellen Spalte der Zeile
subtrahiert. Als Letztes wird der Wert der aktuellen Zeile auf null gesetzt, damit das Gleichungssystem in der Dreiecksform
bleibt.

{\let\clearpage\relax \chapter{Abwägungen}}
\section{Vorteile des Algorithmus}
Der \GA  hat klar den Vorteil, dass er bei Gleichungssystemen mit wenig Koeffizienten
sehr effektiv ist, das heißt er kann die Lösung schnell und einfach berechnen.
Auch ist der Algorithmus in der Programmierung relativ leicht umzusetzen,
da er nach einfachen Schemata funktioniert, bei welchen wenige Schritte benötigt werden.
Ein weiterer Vorteil ist, dass der Algorithmus durch die Pivotisierung eine gute numerische Stabilität bekommt.
\section{Nachteile des Algorithmus}
Zum einen kann es durch die Verwendung vom Datentyp \texttt{double} und die mehrfache Rechnung mit denselben Werten zu Rundungsfehlern kommen.
Zum anderen hat der Algorithmus bei komplizierten Matrizen mit vielen Zeilen und Spalten eine große Laufzeit,
die Worstcase Laufzeit würde aufgrund von doppelten For-Schleifen $ O(n^2) $ betragen.
Außerdem kann der Algorithmus bei vielen Koeffizienten eine
sehr hohe Rechenleistung in Anspruch nehmen.
Ein weiterer Nachteil dieser Implementation ist, dass bei einer Matrix wie in \texttt{beispiel2.txt}, eine Singularität auftritt, es wird -9.223372036854776E14 und sonst zwei
Null-Zeilen ausgegeben, was daran liegen könnte das der Computer aufgrund der vorher gerundeten Zahlen numerisch ungenau rechnet. Dies herauszufinden ist allerdings nicht im Rahmen dieser Arbeit.

{\let\clearpage\relax \chapter{Anwendungsbereiche des Gauß'schen \\ Eliminationsverfahrens in der Informatik}}
Der Algorithmus spielt in der Informatik aufgrund seiner Simplizität in vielen Teilbereichen,
wie der Kryptografie, der Computergrafik oder der Datenanalyse eine tragende Rolle.
Eine Auswahl von Anwendungsbereichen wird im Folgenden umrissen.
\section{Kryptografie}
Das Gauß'sche Eliminationsverfahren kann verwendet werden, um modulare Gleichungen zu lösen,
die in der Kryptografie häufig auftreten.
Zum Beispiel wird das Verfahren bei der Berechnung von Schlüsseln in asymmetrischen Kryptosystemen wie RSA eingesetzt.
Hierbei wird mit dem Gauß-Algorithmus eine Primfaktorzerlegung durchgeführt, die dann
verwendet wird, um Schlüssel in asymmetrischen Kryptosystemen zu verwenden. Vgl. \cite{3} Kapitel 11.4.3.

\section{Computergrafik}
In der 3D-Computergrafik werden 4x4-Matrizen verwendet, um Objekte im Raum zu transformieren.
Diese Matrizen enthalten Informationen über Translationen, Rotationen und Skalierungen von Objekten.
Das Gauß'sche Eliminationsverfahren wird verwendet, um die inverse Matrix zu berechnen, die dann angewendet wird,
um die Transformationsoperationen umzukehren. Sie wird auch dazu benutzt, um Normale von 3D-Objekten zu transformieren,
um sie in eine konsistente Richtung zu bringen, was wichtig für Beleuchtungs- und Schattierungsberechnungen ist. Vgl. \cite{6}.

{\let\clearpage\relax \chapter{Fazit}}
Die lineare Algebra und das Lösen Linearer Gleichungssysteme sind fundamentale Konzepte in der angewandten Mathematik
und finden auch in der Informatik weitreichende Anwendungen. Der \GA  ist eines der elementarsten und faszinierendsten Verfahren
zur Lösung Linearer Gleichungssysteme und hat seit seiner Entdeckung im frühen 19. Jahrhundert große Bedeutung erlangt.
Der mathematische Hintergrund und die Definitionen Linearer Gleichungssysteme sowie des Gaußschen Algorithmus wurden
erläutert sowie eine Implementierungsmöglichkeit des Algorithmus vorgestellt.
Er hat den Vorteil, dass er das Lösen Linearer Gleichungssysteme auf einfachste Weise ermöglicht,
aber es gibt auch Nachteile, wie die Notwendigkeit einer hohen Rechenleistung bei großen Matrizen.
Dennoch sind die Anwendungsmöglichkeiten in der Informatik zahlreich.
Zusammenfassend lässt sich sagen, dass der \GA  ein wichtiges Werkzeug für
Mathematiker und Informatiker ist, um komplexe Probleme zu lösen und praktische Anwendungen zu ermöglichen.


{\let\clearpage\relax\chapter*{Anhang}}
\footnotesize
\lstinputlisting[language=Java, caption=Quellcode des Gauß-Algorithmus]{./Quellcode/Gauss.java}
\printbibliography
\vspace*{\fill}
\vspace*{5pt}
\hrule
\vspace{5pt}
    \textbf{Satz:}
    \LaTeX
\newpage
\includegraphics[width=\textwidth]{Eigenständig.jpeg}
\end{document}
